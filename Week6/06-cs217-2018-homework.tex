\documentclass[12pt,a4]{article}




\newcommand{\handoutdate}{Thursday, 2019-10-30}
\newcommand{\firstduedate}{Monday, 2020-6-1}
\newcommand{\finalduedate}{Thursday, 2020-11-14}





\usepackage{graphicx,amsmath,amssymb, boxedminipage}



%\usepackage{algorithm}
\usepackage[ruled]{algorithm2e}
\usepackage{algpseudocode}
\usepackage{xcolor}
\usepackage{enumerate}
\usepackage{listings}

\newtheorem{theorem}{Theorem}%[section]
\newtheorem{proposition}[theorem]{Proposition}
\newtheorem{lemma}[theorem]{Lemma}
\newtheorem{corollary}[theorem]{Corollary}
\newtheorem{definition}[theorem]{Definition}

\newenvironment{proof}{\paragraph{Proof:}}{\hfill$\square$}
\newenvironment{jie}{\paragraph{Show:}}{\hfill$\square$}


\newcommand{\scalar}[2]{\ensuremath{\langle #1, #2\rangle}}
\newcommand{\floor}[1]{\left\lfloor #1 \right\rfloor}
\newcommand{\ceil}[1]{\left\lceil #1 \right\rceil}
\newcommand{\norm}[1]{\|#1\|}
\newcommand{\pfrac}[2]{\left(\frac{#1}{#2}\right)}
\newcommand{\nth}[1]{#1\textsuperscript{th}}

% \newcommand{\nth}[1]{#1\textsuperscript{th}}
\newcommand{\E}{\mathop{\mathbb{E\/}}}
\newcommand{\N}{\mathbb{N}}

\newcommand{\R}{\mathbb{R}}

\newtheorem{exercise}[theorem]{Exercise}
\newtheorem{exerciseD}[theorem]{*Exercise}
\newtheorem{exerciseDD}[theorem]{**Exercise}

\let\oldexercise\exercise
\renewcommand{\exercise}{\oldexercise\normalfont}

\let\oldexerciseD\exerciseD
\renewcommand{\exerciseD}{\oldexerciseD\normalfont}

\let\oldexerciseDD\exerciseDD
\renewcommand{\exerciseDD}{\oldexerciseDD\normalfont}


\let\implies\Rightarrow
\let\impliedby\Leftarrow
\let\iff\Leftrightarrow
\let\ldots\cdots

\newcommand{\incfig}[2][1]{%
    \def\svgwidth{#1\columnwidth}
    \import{./figures/}{#2.pdf_tex}
}

\newcommand\dif{\,\mathrm{d}}
\newcommand\e{\,\mathrm{e}}
\newcommand\Q{\,\mathbb{Q}}
\newcommand\C{\,\mathbb{C}}
\newcommand\Z{\,\mathbb{Z}}
\newcommand\ep{\,\varepsilon}
\newcommand\F{\,\varphi}
\newcommand\T{\,\mathbb{T}}
\newcommand\HH{\,\mathbb{H}}
 
\begin{document}

\date{}

\title{CS 217 -- Algorithm Design and Analysis \\ 
  \vspace{3mm}
{\large	Shanghai Jiaotong University, Spring 2020\\
}
}
\maketitle

\noindent
Handed out on \handoutdate{}\\
First submission and questions due on \firstduedate{}\\
You will receive feedback from the TA.\\
Final submission due on \finalduedate{}





\newcommand{\rank}{\textnormal{rank}}
\newcommand{\y}{\mathbf{y}}
\renewcommand{\c}{\mathbf{c}}
\newcommand{\x}{\mathbf{x}}
\newcommand{\z}{\mathbf{z}}
\renewcommand{\u}{\mathbf{u}}
\newcommand{\V}{\mathbf{v}}
\newcommand{\val}{\textnormal{val}}

\renewcommand{\a}{\mathbf{a}}

\renewcommand{\b}{\mathbf{b}} 
\newcommand{\zero}{\mathbf{0}}
\newcommand{\rpn}{\mathbb{R}_{\geq 0}}
\newcommand{\sol}{\textup{\textrm{sol}}}
\newcommand{\opt}{\textup{\textrm{opt}}}
\setcounter{section}{5}


\section{Matching LP and Vertex Cover LP}


Let $G=(V,E)$ be a 
graph and consider the Vertex Cover Linear Program $\textrm{VCLP}(G)$:
\begin{align*}
  \textrm{VCLP($G$)}: \qquad
  \begin{array}{lrl}
    \textnormal{minimize} \quad & \multicolumn{2}{l}{\sum_{u \in V} y_u} \\
    \textnormal{subject to} \quad & y_u + y_v  & \geq 1 \quad \forall\ \textnormal{ edges } \{u,v\} \in E\\
     & \y & \geq \mathbf{0}  
  \end{array}
\end{align*}
Every vertex cover of $G$ corresponds to a feasible solution $\y \in \sol(\textrm{VCLP}(G))$, but not 
vice versa. However, every $\y \in \sol(\textrm{VCLP}(G)) \cap \{0,1\}^V$ does.
Let $\tau(G)$ denote the size of a minimum vertex cover of $G$. In class, we showed that
$\tau(G) = \val(\textrm{VCLP}(G))$ for all {\em bipartite} graphs $G$. We achieved this by taking
an arbitrary feasible solution $\y$ and ``shaking'' it until it becomes integral, while making sure
its value does not go up.\\


Next, recall the Matching Linear Program $\textrm{MLP}(G)$:
\begin{align*}
  \textrm{MLP($G$)}: \qquad
  \begin{array}{lrl}
    \textnormal{maximize} \quad & \multicolumn{2}{l}{\sum_{e \in E} x_e} \\
    \textnormal{subject to} \quad & \sum_{e \in E: u \in e} x_e  & \leq 1 \quad \forall\ u \in V \\
     & \x & \geq \mathbf{0}  
  \end{array}
\end{align*}
Every matching of $G$ corresponds to a feasible solution $\x \in \sol(\textrm{MLP}(G))$, but not 
vice versa. However, every $\x \in \sol(\textrm{MLP}(G)) \cap \{0,1\}^E$ does.

\begin{exercise}
   Let $\nu(G)$ denote the size of a maximum matching of $G$. 
   Obviously, $ \val(\textrm{MLP}(G)) \geq \nu(G)$ for all graphs.
   Show that
   $\nu(G) = \val(\textrm{MLP}(G))$ for all {\em bipartite} graphs $G$.
\end{exercise}
\begin{proof}
	We just need to prove that $int-val(MLP(G)) = val(MLP(G))$, since $val(MLP(G)) \ge \nu(G)$,
    we do this following the proof for $VCLP(G)$ in class, the idea is to get int value solution
    through modifying a non-int value solution a bit.

    Suppose we already have a solution $\mathbf{x}$, 
    and denote $\sum_{e \in E, u \in e}x_e = y_u, \forall u \in V$, we want to change $\mathbf{x}$ to int value,
    and meanwhile $\mathbf{y}$ becomes int value.

    At each step, those edges with $x_e= 0$ could just be ignored, while those edges with $x_e = 1$ and its both ends
    could be ignored. In the following situations, we just don't take these edges and vertices into account.
    \begin{enumerate}
        \item 
            For two vertices $u, v$ with non integral $y_u, y_v$, suppose there exits a path $e_1, e_2, ... e_m$, where $m$ is
            odd. Let $d = min(1 - y_u, 1-y_v, 1-x_{e_1}, x_{e_2}, ..., 1- x_{e_k})$. Next we modify edges as following.
            For those $e_i, i odd$, let $e_i = e_i + d$, and for those $e_i, i even$, let $e_i = e_i - d$. By doing this,
            all constraints remains, and at least one of $y_u, y_v, x_{e_1}, ... x_{e_k}$ becomes integral. In all, $val(MLP(G))$ increases by d. 
        \item
            Similar to the situation above, except that the path has an even size.
            Let $d = min(1 - y_u, y_v, 1-x_{e_1}, x_{e_2}, ..., 1- x_{e_{k - 1}}, x_{e_k})$. Next we modify edges as the above situation.
            By doing this, all constraints still remains, 
            and at least one of $y_u, y_v, x_{e_1}, ... x_{e_k}$ becomes integral. In all, $val(MLP(G))$ remains unchanged.
        \item 
            If $y_u$ is non integral but $y_v$ integral, we just find all vertices connected with $u, v$. Denote the left side as $L$
            and the right side as $R$, We shall have $\sum_{u \in L} y_u = \sum_{u \in R} y_u$. But the left side is non integral while 
            the right side is integral, leading to a contradiction.
        \item 
            If all $y_u$ are integral, but some $x_e$ may still remain non integral. To solve this,
            we find one of the non integral edge $e_1$, with $u$ as one of its end. Since $y_u$ integral, there exists another $e_2$
            non integral connected to $u$. Continue this, we will find a cycle $e_1,...,e_k$ with even size since $G$ is bipartite.
            Let $d = min(1-x_{e_1}, x_{e_2}, ... 1-x_{e_{k - 1}}, x_{e_k}$, do the updates as the above updates.
            We can see that $\mathbf{y}$ remains, while at least one $x_e$ becomes integral, and $val(MLP(G))$ remains.
    \end{enumerate}

    In the above situations, at least one $y_u$ or one $x_e$ becomes integral. Since both $\mathbf{x}, \mathbf{y}$ are finite, 
    this procedure will end. And since $x_e$ becomes all integral, the original solution becomes an int value solution while not making 
    $\sum_{e \in E}x_e$ worse. Thus we know $\nu(G) = val(MLP(G))$
\end{proof}
\begin{exercise}
  We know that $\nu(G) = \tau(G)$ for all bipartite graphs (K\H{o}nig's Theorem) and
  $\nu(G) \leq \tau(G)$ for all graphs (since every matched edge must be covered
  by a distinct vertex). Show that $\tau(G) \leq 2 \, \nu(G)$ for all graphs $G$.   
\end{exercise}
\begin{proof}
		Assume that $\alpha$ be the size of maximum independent set, and $\beta$ be the size of a minimum vertex cover of G. Let $U$ be the set of vertex cover, and $V/U$ must be an independent set, because if there is any edge between $S/U$, $U$ is not a cover, contradiction. So we have $\alpha + \beta=|V|(1)$. 

		Suppose $(a_1,b_1),(a_2,b_2),\cdots,(a_{\nu(G)},b_{\nu(G)})$ be a maximum match. Let $W$ be the vertex set that is not in the maximum match. Obviously, $W$ is an independent set. By definition, $2\nu(G)+|W|= |V|$. Because W is an independent set, $|W|\le \alpha=|V|-\tau(G)$. So we get $\tau(G)\le 2\nu(G)$.
\end{proof}
\begin{exercise}
   Show that $\tau(G) \leq 2\, \opt(\textrm{VCLP}(G))$ for all graphs $G$ (including non-bipartite graphs).
\end{exercise}
\begin{proof}
	By the result of last exercise
	\[
			\tau\left( G \right) \le 2 \nu\left( G \right) 
	.\] 
	And $\nu\left( G \right) \le \opt\left( VCLP\left( G \right)  \right) $, the result is obvious.
\end{proof}

\begin{exercise}
	
\end{exercise}
\begin{proof}
		(1) 
		Consider such a graph: $\left| V \right| =4,	E = E(K_4) / e_{1,2}$, obviously it's not bipartite.
		We have
		\[
				\nu\left( G \right) = 2 = \nu_{f}\left( G \right) = \tau_{f}\left( G \right) =\tau\left( G \right) 
		.\] 
		(2) Let $G = K_4$, then 
		\[
				\nu\left( G \right) =2 = \nu_{f}\left( G \right) =\tau_{f}\left( G \right) < \tau\left( G \right) = 3
		.\] 
		(3) Since $G$ is VCLP exact. A MVC $Y$ also corresponds to a optimal solution in VCLP. If 
		$e=\left( u_0,v_0 \right) \in Y$. We have 
		\[
				y_{u_0} = y_{v_0} = 1 \implies y_{u_0} + y_{v_0} = 2 > 1
		.\] 
		Hence we wouldn't use this inequality $y_{u_0} + y_{v_0} \ge 1$ when converting the VCLP to its dual,
		therefore the respective coefficient $x_{e} = 0$. More specifically:
		\[
				\sum _{u \in V} y_{u}\ge  \sum_{e\in E} x_e \left( y_u + y_v \right) \ge \sum _{e\in E} x_e
		.\] 
		As $\min \sum_{u\in V} y_u= \max \sum_{e\in E} x_e$. 
		These inequalities are all tight, which leads that 
		\[
				x_e = 0 \text{ or } y_{u} + y_{v} = 1
		.\] 
		If $y_u + y_v > 1, x_e \neq 0$, the inequality is not tight, leads to a contradiction.
		\\(4)First we prove there is a matching of size $s=\left| Y \right| $.
		Consider all the vertexes $v_1,v_2,\ldots,v_s\in Y$. If $\exists i\neq j$ s.t.
		$e=(v_i,v_j)\in E$, then $x_e = 0$. We can assert that  $\exists u_i\neq u_j$ s.t.
		\[
				\left( v_i,u_i \right) ,\left( v_j,u_j \right) \in E
		.\]
		Plus $u_i\neq v_j, u_j\neq v_i$, otherwise we can obtain a smaller vertex cover
		by removing one of $v_i,v_j$. And otherwise  $v_i$ has no neighbors in  $Y$, 
		we just choose an arbitrary neighbor(since it's MVC, it must have a neighbor), 
		this neighbor cannot be one of  $u_i$ 
		mentioned above, otherwise there is smaller vertex cover by removing this  $v_i$.
		Hence we get a matching  of size $\left| Y \right| $
		\[
		\max \sum_{e\in E} x_e \ge  \left| Y \right| 
		.\] 
		While 
		\[
				\max \sum_{e\in E} x_e \le \tau_{f}\left( G \right)  = \left| Y \right| 
		.\] 
		It follows that 
		\[
				\nu\left( G \right) = \nu_{f}\left( G \right) 
		.\] 
		Hence $G$ is MLP-exact too.
\end{proof}
\end{document}
