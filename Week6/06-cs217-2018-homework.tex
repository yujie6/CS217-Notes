\documentclass[12pt,a4]{article}




\newcommand{\handoutdate}{Thursday, 2019-10-30}
\newcommand{\firstduedate}{Monday, 2020-6-1}
\newcommand{\finalduedate}{Thursday, 2020-11-14}





\usepackage{graphicx,amsmath,amssymb,amsthm, boxedminipage}



\usepackage{algorithm}
\usepackage{algpseudocode}


\newtheorem{theorem}{Theorem}%[section]
\newtheorem{proposition}[theorem]{Proposition}
\newtheorem{lemma}[theorem]{Lemma}
\newtheorem{corollary}[theorem]{Corollary}
\newtheorem{definition}[theorem]{Definition}



\newcommand{\scalar}[2]{\ensuremath{\langle #1, #2\rangle}}
\newcommand{\floor}[1]{\left\lfloor #1 \right\rfloor}
\newcommand{\ceil}[1]{\left\lceil #1 \right\rceil}
\newcommand{\norm}[1]{\|#1\|}
\newcommand{\pfrac}[2]{\left(\frac{#1}{#2}\right)}
\newcommand{\nth}[1]{#1\textsuperscript{th}}

% \newcommand{\nth}[1]{#1\textsuperscript{th}}
\newcommand{\E}{\mathop{\mathbb{E\/}}}
\newcommand{\N}{\mathbb{N}}

\newcommand{\R}{\mathbb{R}}

\newtheorem{exercise}[theorem]{Exercise}
\newtheorem{exerciseD}[theorem]{*Exercise}
\newtheorem{exerciseDD}[theorem]{**Exercise}

\let\oldexercise\exercise
\renewcommand{\exercise}{\oldexercise\normalfont}

\let\oldexerciseD\exerciseD
\renewcommand{\exerciseD}{\oldexerciseD\normalfont}

\let\oldexerciseDD\exerciseDD
\renewcommand{\exerciseDD}{\oldexerciseDD\normalfont}


 
\begin{document}

\date{}

\title{CS 217 -- Algorithm Design and Analysis \\ 
  \vspace{3mm}
{\large	Shanghai Jiaotong University, Fall 2019\\
}
}
\maketitle

\noindent
Handed out on \handoutdate{}\\
First submission and questions due on \firstduedate{}\\
You will receive feedback from the TA.\\
Final submission due on \finalduedate{}





\newcommand{\rank}{\textnormal{rank}}
\newcommand{\y}{\mathbf{y}}
\renewcommand{\c}{\mathbf{c}}
\newcommand{\x}{\mathbf{x}}
\newcommand{\z}{\mathbf{z}}
\renewcommand{\u}{\mathbf{u}}
\newcommand{\V}{\mathbf{v}}
\newcommand{\val}{\textnormal{val}}

\renewcommand{\a}{\mathbf{a}}

\renewcommand{\b}{\mathbf{b}} 
\newcommand{\zero}{\mathbf{0}}
\newcommand{\rpn}{\mathbb{R}_{\geq 0}}
\newcommand{\sol}{\textup{\textrm{sol}}}
\newcommand{\opt}{\textup{\textrm{opt}}}
\setcounter{section}{5}


\section{Matching LP and Vertex Cover LP}


Let $G=(V,E)$ be a 
graph and consider the Vertex Cover Linear Program $\textrm{VCLP}(G)$:
\begin{align*}
  \textrm{VCLP($G$)}: \qquad
  \begin{array}{lrl}
    \textnormal{minimize} \quad & \multicolumn{2}{l}{\sum_{u \in V} y_u} \\
    \textnormal{subject to} \quad & y_u + y_v  & \geq 1 \quad \forall\ \textnormal{ edges } \{u,v\} \in E\\
     & \y & \geq \mathbf{0}  
  \end{array}
\end{align*}
Every vertex cover of $G$ corresponds to a feasible solution $\y \in \sol(\textrm{VCLP}(G))$, but not 
vice versa. However, every $\y \in \sol(\textrm{VCLP}(G)) \cap \{0,1\}^V$ does.
Let $\tau(G)$ denote the size of a minimum vertex cover of $G$. In class, we showed that
$\tau(G) = \val(\textrm{VCLP}(G))$ for all {\em bipartite} graphs $G$. We achieved this by taking
an arbitrary feasible solution $\y$ and ``shaking'' it until it becomes integral, while making sure
its value does not go up.\\


Next, recall the Matching Linear Program $\textrm{MLP}(G)$:
\begin{align*}
  \textrm{MLP($G$)}: \qquad
  \begin{array}{lrl}
    \textnormal{maximize} \quad & \multicolumn{2}{l}{\sum_{e \in E} x_e} \\
    \textnormal{subject to} \quad & \sum_{e \in E: u \in e} x_e  & \leq 1 \quad \forall\ u \in V \\
     & \x & \geq \mathbf{0}  
  \end{array}
\end{align*}
Every matching of $G$ corresponds to a feasible solution $\x \in \sol(\textrm{MLP}(G))$, but not 
vice versa. However, every $\x \in \sol(\textrm{MLP}(G)) \cap \{0,1\}^E$ does.

\begin{exercise}
   Let $\nu(G)$ denote the size of a maximum matching of $G$. 
   Obviously, $ \val(\textrm{MLP}(G)) \geq \nu(G)$ for all graphs.
   Show that
   $\nu(G) = \val(\textrm{MLP}(G))$ for all {\em bipartite} graphs $G$.
\end{exercise}
\begin{proof}
		
\end{proof}
\begin{exercise}
  We know that $\nu(G) = \tau(G)$ for all bipartite graphs (K\H{o}nig's Theorem) and
  $\nu(G) \leq \tau(G)$ for all graphs (since every matched edge must be covered
  by a distinct vertex). Show that $\tau(G) \leq 2 \, \nu(G)$ for all graphs $G$.   
\end{exercise}
\begin{proof}
		
\end{proof}
\begin{exercise}
   Show that $\tau(G) \leq 2\, \opt(\textrm{VCLP}(G))$ for all graphs $G$ (including non-bipartite graphs).
\end{exercise}
\begin{proof}
	By the result of last exercise
	\[
			\tau\left( G \right) \le 2 \nu\left( G \right) 
	.\] 
	And $\nu\left( G \right) \le \opt\left( VCLP\left( G \right)  \right) $, the result is obvious.
\end{proof}

\textbf{Basic Solutions.} Recall our definition of basic solutions.
Let $P$ be the following linear program.
\begin{align*}
  P: \qquad
  \begin{array}{lrl}
    \textnormal{maximize} \quad & \multicolumn{2}{l}{\mathbf{c}^T \x} \\
    \textnormal{subject to} \quad & A\x & \leq \b \ 
  \end{array}
\end{align*}
where we translated the constraint $\x \geq 0$ into $n$ constraints
$-x_i \leq 0$ and integrated them into $A$, so 
the $n$ last rows of $A$ form the 
negative identity matrix $-I_n$. We introduce some notation: $\a_i$ is the $\nth{i}$ row of $i$;
for $I \subseteq [m+n]$ let $A_I$ be the matrix
consisting of the rows $\a_i$ for $i \in I$.
\begin{definition}
  For $\x \in \R^n$ let $I(\x) := \{i \in [m+n] \ | \ \a_i \x = b_i\}$
  be the set of indices of the constraints that are ``tight'', i.e.,
  satisfied with equality (we include non-negativity constraints here).  We call $\x \in \R^n$ a {\em basic point}
  if $\rank(A_{I(x)}) = n$.  If $\x$ is a basic point and feasible, we
  call it a {\em basic feasible solution} or simply a {\em basic
    solution}.
\end{definition}
We can define the same concept for minimization programs. \\

We say a set $C \subseteq V$ is a minimal vertex cover of $G = (V,E)$ if 
(1) it is a vertex cover and (2) it is minimal, i.e., for every $u \in C$ the set
$C \setminus \{u\}$ is not a vertex cover anymore.

\begin{exercise}

\end{exercise}
\begin{proof}
		(1) 
		Consider such a graph: $\left| V \right| =4,	E = E(K_4) / e_{1,2}$, obviously it's not bipartite.
		We have
		\[
				\nu\left( G \right) = 2 = \nu_{f}\left( G \right) = \tau_{f}\left( G \right) =\tau\left( G \right) 
		.\] 
		(2) Let $G = K_4$, then 
		\[
				\nu\left( G \right) =2 = \nu_{f}\left( G \right) =\tau_{f}\left( G \right) < \tau\left( G \right) = 3
		.\] 
		(3) Since $G$ is VCLP exact. A MVC $Y$ also corresponds to a optimal solution in VCLP. If 
		$e=\left( u_0,v_0 \right) \in Y$. We have 
		\[
				y_{u_0} = y_{v_0} = 1 \implies y_{u_0} + y_{v_0} = 2 > 1
		.\] 
		Hence we wouldn't use this inequality $y_{u_0} + y_{v_0} \ge 1$ when converting the VCLP to its dual,
		therefore the respective coefficient $x_{e} = 0$. More specifically:
		\[
				\sum _{u \in V} y_{u}\ge  \sum_{e\in E} x_e \left( y_u + y_v \right) \ge \sum _{e\in E} x_e
		.\] 
		As $\min \sum_{u\in V} y_u= \max \sum_{e\in E} x_e$. 
		These inequalities are all tight, which leads that 
		\[
				x_e = 0 \text{ or } y_{u} + y_{v} = 1
		.\] 
		If $y_u + y_v > 1, x_e \neq 0$, the inequality is not tight, leads to a contradiction.
		\\(4) Since there is a matching of size $\left| Y \right| $.
		\[
		\max \sum_{e\in E} x_e \ge  \left| Y \right| 
		.\] 
		While 
		\[
				\max \sum_{e\in E} x_e \le \tau_{f}\left( G \right)  = \left| Y \right| 
		.\] 
		It follows that 
		\[
				\nu\left( G \right) = \nu_{f}\left( G \right) 
		.\] 
		Hence $G$ is MLP-exact too.
\end{proof}
\end{document}
