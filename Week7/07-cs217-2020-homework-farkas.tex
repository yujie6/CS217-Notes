\documentclass[12pt,a4]{article}

\usepackage{xcolor}


\newcommand{\handoutdate}{Friday, 2020-06-05}
\newcommand{\firstduedate}{Thursday, 2020-06-12}
\newcommand{\finalduedate}{Thursday, 2020-06-19}





\usepackage{graphicx,amsmath,amssymb, boxedminipage}



%\usepackage{algorithm}
\usepackage[ruled]{algorithm2e}
\usepackage{algpseudocode}
\usepackage{xcolor}
\usepackage{enumerate}
\usepackage{listings}

\newtheorem{theorem}{Theorem}%[section]
\newtheorem{proposition}[theorem]{Proposition}
\newtheorem{lemma}[theorem]{Lemma}
\newtheorem{corollary}[theorem]{Corollary}
\newtheorem{definition}[theorem]{Definition}

\newenvironment{proof}{\paragraph{Proof:}}{\hfill$\square$}
\newenvironment{jie}{\paragraph{Show:}}{\hfill$\square$}


\newcommand{\scalar}[2]{\ensuremath{\langle #1, #2\rangle}}
\newcommand{\floor}[1]{\left\lfloor #1 \right\rfloor}
\newcommand{\ceil}[1]{\left\lceil #1 \right\rceil}
\newcommand{\norm}[1]{\|#1\|}
\newcommand{\pfrac}[2]{\left(\frac{#1}{#2}\right)}
\newcommand{\nth}[1]{#1\textsuperscript{th}}

% \newcommand{\nth}[1]{#1\textsuperscript{th}}
\newcommand{\E}{\mathop{\mathbb{E\/}}}
\newcommand{\N}{\mathbb{N}}

\newcommand{\R}{\mathbb{R}}

\newtheorem{exercise}[theorem]{Exercise}
\newtheorem{exerciseD}[theorem]{*Exercise}
\newtheorem{exerciseDD}[theorem]{**Exercise}

\let\oldexercise\exercise
\renewcommand{\exercise}{\oldexercise\normalfont}

\let\oldexerciseD\exerciseD
\renewcommand{\exerciseD}{\oldexerciseD\normalfont}

\let\oldexerciseDD\exerciseDD
\renewcommand{\exerciseDD}{\oldexerciseDD\normalfont}


\let\implies\Rightarrow
\let\impliedby\Leftarrow
\let\iff\Leftrightarrow
\let\ldots\cdots

\newcommand{\incfig}[2][1]{%
    \def\svgwidth{#1\columnwidth}
    \import{./figures/}{#2.pdf_tex}
}

\newcommand\dif{\,\mathrm{d}}
\newcommand\e{\,\mathrm{e}}
\newcommand\Q{\,\mathbb{Q}}
\newcommand\C{\,\mathbb{C}}
\newcommand\Z{\,\mathbb{Z}}
\newcommand\ep{\,\varepsilon}
\newcommand\F{\,\varphi}
\newcommand\T{\,\mathbb{T}}
\newcommand\HH{\,\mathbb{H}}
 
\begin{document}

\date{}

\title{CS 217 -- Algorithm Design and Analysis \\ 
  \vspace{3mm}
{\large	Shanghai Jiaotong University, Spring 2020\\
}
}
\maketitle

\noindent
Handed out on \handoutdate{}\\
First submission and questions due on \firstduedate{}\\
You will receive feedback from the TA.\\
Final submission due on \finalduedate{}





\newcommand{\rank}{\textnormal{rank}}
\newcommand{\y}{\mathbf{y}}
\renewcommand{\c}{\mathbf{c}}
\newcommand{\x}{\mathbf{x}}
\newcommand{\z}{\mathbf{z}}
\renewcommand{\u}{\mathbf{u}}
\newcommand{\V}{\mathbf{v}}

\renewcommand{\a}{\mathbf{a}}

\renewcommand{\b}{\mathbf{b}} 
\newcommand{\zero}{\mathbf{0}}
\newcommand{\rpn}{\mathbb{R}_{\geq 0}}
\newcommand{\sol}{\textup{\textrm{sol}}}
\newcommand{\opt}{\textup{\textrm{opt}}}
\setcounter{section}{6}


\section{Farkas Lemma and LP Duality}

\subsection{Different Versions of Farkas Lemma}

In the following, let $A \in \R^{m \times n}$ and $\b \in \R^m$, and let 
$\x = (x_1,\dots,x_n)^T$ be a column vector of $n$ variables and 
$\y = (y_1, \dots,y_m)$ be a row vector of $m$ variables.

\begin{exercise}
 Show that the three versions of Farkas Lemma presented in class are all equivalent (I actually did not present
 the third version in class):
 \begin{align}
   ( \neg \exists \x : \, A \x \leq \b ) \, & \Longleftrightarrow 
    ( \exists \y \geq \zero : \, \y^T A = \zero, \,     \y^T \b < 0 ) \ . \\
      ( \neg \exists \x \geq \zero : \, A \x \leq \b ) \, & \Longleftrightarrow 
    ( \exists \y \geq \zero : \, \y^T A \geq \zero, \,  \y^T \b < 0 ) \ . \\
   ( \neg \exists \x \geq \zero : \, A \x = \b ) \, & \Longleftrightarrow 
    ( \exists \y \begingroup \color{white} \geq \zero \endgroup : \, \y^T A \geq \zero, \,  \y^T \b < 0 ) \ .
 \end{align}
  Note that the direction ``$\Longleftarrow$'' is easy in each case. 
  We will show the ``$\Longrightarrow$'' of (1) in class using a technique called {\em Fourier-Motzkin Elimination}. 
  This exercise is actually not that hard. The hardest part is keeping track of what you 
  want to prove and what you can assume.
\end{exercise}


\subsection{A Linear Program for, well, for what?}




Let $G = (V,E)$ be a directed graph, $s,t \in V$,  and $c: E \rightarrow \mathbf{R}^+$ be a cost 
function. We want to find an $s-t$-flow $f$ of value $1$. Every edge $e$ generates cost $f(e) \cdot c(e)$, and we want to minimize the overall cost. There are no capacity constraints.
We can easily write this as a linear program MCF (Minimum Cost Flow):
\begin{align*}
  \textrm{MCF}(G,s,t,c): \qquad
  \begin{array}{ll}
    \textnormal{minimize} \quad & \multicolumn{1}{l}{\sum_{e \in E} c(e) f(e)} \\
    \\
    \textnormal{subject to} \quad & \sum_{v \in V} f(v,t)  = 1 \\
					        & \sum_{u \in V} f(u,v) - \sum_{w \in V} f(v,w)  = 0  \quad \forall\ v \in V  \setminus \{s,t\}\\
					        \\
     & f(e)  \geq 0 \ \forall \ e \in E 
  \end{array}
\end{align*}
Note that we have $m$ variables, one variable $f(e)$ for each edge $e$.
The first constraint says that the value of the flow should be 1. The other constraints say that 
the inflow at $v$ should equal the outflow.

\begin{exercise}
   Let $d$ be the shortest path distance from $s$ to $t$ in the directed graph $G$, where distance
   means sum of the $c(e)$ along the path. Show that $\opt(MCF) = d$.
   \textbf{Hint.} Make sure you show both $\leq$ and $\geq$.
\end{exercise}

\begin{proof}
To prove $opt(MCF) \le d$, we just need to prove that the shortest path is a solution to $MCF$.
    We set $f(e) = 1$ along all edges in the shortest path, since there is only one path with flow $1$,
    The constraints are obviously satisfied. So it is a solution of $MCF$, and its value is $1 * d = d$,
    so $opt(MCF) \le d$

    To prove $opt(MCF) \ge d$, we need to prove that all solutions of $MCF$ is not better than $d$.
    We try to improve the value of all possible solutions to $d$.
    
    Suppose we have a solution with $x$ different $s-t$ path. Define $b(path)$ be the smallest flow in all
    edges of $path$, $d(path)$ be the length of $path$. Let $sp$ be the shortest $s-t$ path. We do as follows, choose any path $p$ besides $sp$,
    put $b(p)$ units of flow on $p$ to $sp$. Repeat it until there is only $1$ unit flowing through $sp$.

    We need to show in each turn, $val(MCF)$ is not worse than previous and no constraints are broke. We fist prove 
    $val(MCF)$ is not worse. In each turn, $val(MCF)' = val(MCF) + d * b(p) - d(p) * b(p)$, $d \le d(p)$, so $val(MCF)' \le val(MCF)$.
    As for the constraints, inflow of $t$ remains to be 1 since we just move $b(p)$ units between two different paths.
    Flow constraints remains since we modify the flow in one path, which means we move inflow and outflow of a single 
    vertex at the same time. Since we only have $x$ different $s-t$ path, and the flow value on each path is finite,
    the process terminates. So $val(MCF) \ge d$

    In all $val(MCF) = d$
\end{proof}

\begin{exercise}
    Write down the dual of MCF. This will be a maximization problem. Don't use any matrix notation.
\end{exercise}

\begin{proof}
We introduce a dual coefficient $g_v, v \in V$. The dual program is:
    \begin{itemize}
        \item Maximize $g_t$, subject to:
        \item $g_v - g_u \le c(u, v), \forall (u,v) \in E$
        \item $g_v \in \mathbb{R}, v \in V$
    \end{itemize}
\end{proof}

\begin{exercise}
   Interpret the dual. Show that it is the LP formulation of a ``natural'' maximization problem on $G$.
\end{exercise}

\begin{proof}
If we set $g_s = 0$, then the $g_v$ can be thought as the cost of some $s-t$-path.
    Since each edge $(u,v)$ must satisfy $g_v - g_u \le c(u,v))$, we can not just choose the
    maximal $s-t$-path as solution. Under this constraint, we can see that the solution
    must at first be a \textbf{safe} path, so the program is actually the shortest path problem.
\end{proof}

\begin{exercise}
  Describe an optimal solution of the dual program.
\end{exercise}

\begin{proof}
    The optimal solution is the shortest $s-t$-path $sp$, and for each vertex along the path, we must 
    set $g_v = g_u$ for $(u,v), u \in sp, v \notin sp$ accordingly to $g_v-g_u$ constraints.
\end{proof}
   
   
   
  


\end{document}