\documentclass[UTF8]{ctexart} 
\usepackage{amsmath} 
\usepackage{graphicx}
\usepackage{pythonhighlight}
\usepackage{indentfirst}
\usepackage{amsfonts}
\usepackage{graphicx}
\usepackage{subfig}

% \usepackage{algorithm}
\usepackage[]{algorithm2e}
\begin{document} 

\title{Homework 5}
\author{Ji Jiabao}
\maketitle

\subsection*{Exer.1}
    \begin{enumerate}
        \item 
            $\Leftarrow:$ 
            Suppose $\exists \mathbf{x} : A\mathbf{x}\le\mathbf{b}$,
            Then $0 = 0 \cdot \mathbf{x}=(\mathbf{y}^TA)\mathbf{x} \le \mathbf{y}^T\mathbf{b} < 0$,
            $0 < 0$, contradiction, so $\neg \exists \mathbf{x} : A\mathbf{x}\le\mathbf{b}$
        \item 
            $\Leftarrow:$
            Similar to the proof in case (1), $0 \le 0 \cdot \mathbf{x} \le (\mathbf{y}^TA)\mathbf{x} \le \mathbf{y}^T\mathbf{b} < 0$
            0 < 0, contradiction
        \item 
            $\Leftarrow:$
            Similar to the proof above, $0 \le 0 \cdot \mathbf{x} \le (\mathbf{y}^TA)\mathbf{x} = \mathbf{y}^T\mathbf{b} < 0$,
            0 < 0, contradiction
    \end{enumerate}

\subsection*{Exer.2}
    To prove $opt(MCF) \le d$, we just need to prove that the shortest path is a solution to $MCF$.
    We set $f(e) = 1$ along all edges in the shortest path, since there is only one path with flow $1$,
    The constraints are obviously satisfied. So it is a solution of $MCF$, and its value is $1 * d = d$,
    so $opt(MCF) \le d$

    To prove $opt(MCF) \ge d$, we need to prove that all solutions of $MCF$ is not better than $d$.
    We try to improve the value of all possible solutions to $d$.
    
    Suppose we have a solution with $x$ different $s-t$ path. Define $b(path)$ be the smallest flow in all
    edges of $path$, $d(path)$ be the length of $path$. Let $sp$ be the shortest $s-t$ path. We do as follows, choose any path $p$ besides $sp$,
    put $b(p)$ units of flow on $p$ to $sp$. Repeat it until there is only $1$ unit flowing through $sp$.

    We need to show in each turn, $val(MCF)$ is not worse than previous and no constraints are broke. We fist prove 
    $val(MCF)$ is not worse. In each turn, $val(MCF)' = val(MCF) + d * b(p) - d(p) * b(p)$, $d \le d(p)$, so $val(MCF)' \le val(MCF)$.
    As for the constraints, inflow of $t$ remains to be 1 since we just move $b(p)$ units between two different paths.
    Flow constraints remains since we modify the flow in one path, which means we move inflow and outflow of a single 
    vertex at the same time. Since we only have $x$ different $s-t$ path, and the flow value on each path is finite,
    the process terminates. So $val(MCF) \ge d$

    In all $val(MCF) = d$

\subsection*{Exer.3}
    We introduce a dual coefficient $g_v, v \in V$. The dual program is:
    \begin{itemize}
        \item Maximize $g_t$, subject to:
        \item $g_v - g_u \le c(u, v), \forall (u,v) \in E$
        \item $g_v \in \mathbb{R}, v \in V$
    \end{itemize}

\subsection*{Exer.4}
    If we set $g_s = 0$, then the $g_v$ can be thought as the cost of some $s-t$-path.
    Since each edge $(u,v)$ must satisfy $g_v - g_u \le c(u,v))$, we can not just choose the
    maximal $s-t$-path as solution. Under this constraint, we can see that the solution
    must at first be a \textbf{safe} path, so the program is actually the shortest path problem.
\subsection*{Exer.5}
    The optimal solution is the shortest $s-t$-path $sp$, and for each vertex along the path, we must 
    set $g_v$ for $(u,v), u \in sp, v \notin sp$ accordingly to $g_v-g_u$ constraints.
\end{document}

